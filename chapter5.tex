% !TEX root = main.tex

\chapter{Weakly Supervised Dense Video Captioning with Pretraining}
\label{chap:pws_dvc}

\section{The Challenge of Limited Annotations}

The advancement of dense video captioning has been significantly constrained by the fundamental challenge of limited annotated datasets, which poses critical obstacles for developing robust and generalizable models. While recent progress in parallel processing approaches~\cite{Wang2021-zi,Choi2022-cu} has addressed architectural bottlenecks and error propagation issues, these methods continue to rely heavily on fully supervised learning with expensive temporal annotations that require precise event boundaries and corresponding natural language descriptions.

The scarcity of densely annotated video datasets presents a multifaceted challenge that extends beyond simple data quantity. Current datasets such as ActivityNet Captions~\cite{Krishna2017-pw} and YouCook2~\cite{Zhou2018-eq} contain only 20,000 and 2,000 videos respectively, which pales in comparison to the millions of video-text pairs available in large-scale clip-level datasets~\cite{Miech2019-hk,Bain2021-si,Xu2016-ti}. The annotation process for dense video captioning requires human annotators to precisely identify temporal boundaries for multiple overlapping events and generate corresponding captions, making it substantially more expensive and time-consuming than clip-level video captioning tasks~\cite{Chen2011-ai,Wang2019-tk}.

\textbf{Annotation Dependency in Current Approaches.}
Despite architectural innovations in recent dense video captioning methods, including parallel processing frameworks~\cite{Wang2021-zi,Choi2022-cu} and transformer-based approaches~\cite{Zhou2018-zu,Lin2022-wi}, the fundamental reliance on dense temporal annotations remains unchanged. These methods require ground-truth event boundaries and caption pairs for effective training, creating a significant bottleneck for scaling to larger and more diverse video collections. Even sophisticated approaches that employ multi-modal fusion~\cite{Iashin2020-ln,Rahman2019-rp} or hierarchical processing~\cite{Deng2021-qd} cannot circumvent this annotation dependency, limiting their practical applicability in real-world scenarios where extensive manual annotation is prohibitive.

The annotation challenge becomes particularly acute when considering the nondeterministic nature of event interpretation illustrated in Figure~\ref{fig:nondeterminism}. Different annotators may identify distinct temporal boundaries and generate diverse captions for identical video content, introducing variability that compounds the difficulty of creating consistent training datasets~\cite{Summers2021-mz}. This subjectivity in human annotation not only increases the cost of dataset creation but also raises questions about the reliability and generalizability of models trained on such data.

\begin{figure}[t]
    \centering
    \includegraphics[width=\columnwidth]{figures/pws_introduction_nondeterminism}
    \caption{Example of dense video captioning for a video extracted from the ActivityNet Captions dataset.
        Note the different definitions of events and captions for the same video, highlighting the nondeterministic nature of human annotation.}
    \label{fig:nondeterminism}
\end{figure}

\textbf{Limitations of Current Pretraining Approaches.}
Recent efforts to address annotation scarcity through large-scale pretraining~\cite{Yang2023-fm,Zhang2022-ni,Seo2022-ok} have shown promise but face inherent limitations. Methods like Vid2Seq~\cite{Yang2023-fm} leverage narrated video datasets with automatically transcribed speech, but these approaches suffer from temporal misalignment between narration and visual events, particularly in videos with limited or no speech content. Similarly, approaches that utilize automatically generated pseudo-events~\cite{Zhang2022-ni} often produce noisy supervision signals that may not transfer effectively to downstream dense captioning tasks.

The fundamental challenge lies in bridging the gap between abundant clip-level video-text datasets and the sparse event-level annotations required for dense video captioning. While clip-level datasets contain millions of video-caption pairs that could potentially inform dense captioning models, existing approaches have struggled to effectively leverage this wealth of data due to the mismatch between clip-level and event-level supervision signals.

\textbf{Promise of Weakly Supervised Learning.}
Weakly supervised approaches~\cite{Duan2018-qf,Chen2021-sv,Rahman2019-rp} offer a promising alternative by learning from video-level captions without requiring explicit temporal annotations. However, existing weakly supervised methods face a critical limitation: their performance is heavily dependent on the quality of the captioning module, creating a chicken-and-egg problem where poor captions lead to ineffective localization, which in turn degrades caption quality.

The key insight motivating our approach is that the abundance of high-quality clip-level video captioning datasets~\cite{Xu2016-ti,Bain2021-si,Wang2019-tk} can be strategically leveraged to pretrain robust captioning modules that subsequently enable effective weakly supervised dense video captioning. Unlike fully supervised methods that require expensive event-level annotations, or existing pretraining approaches that rely on noisy automatically generated data, our strategy harnesses the wealth of human-annotated clip-level datasets to bootstrap dense video captioning capabilities.

\begin{figure}[t]
    \centering
    \includegraphics[width=\columnwidth]{figures/pws_Introduction_approach}
    \caption{The approach of the proposed method. Our technique addresses annotation scarcity by pretraining the event captioning module on abundant clip-level datasets, then fine-tuning through weakly supervised learning.
        CL and EC stand for caption localization and event captioning, respectively.}
    \label{fig:approach}
\end{figure}

\textbf{Motivation for PWS-DVC.}
Our proposed PWS-DVC framework, illustrated in Figure~\ref{fig:approach}, addresses the annotation challenge through a two-stage approach that maximizes the utilization of available data while minimizing dependence on expensive temporal annotations. By pretraining the event captioning module on large-scale clip-level datasets and subsequently fine-tuning through weakly supervised learning, PWS-DVC enables effective dense video captioning without requiring the extensive event-level annotations that constrain existing approaches.

This strategy offers several key advantages: (1) it leverages the vast amount of clip-level video-text data already available, (2) it reduces the annotation burden by eliminating the need for precise temporal boundaries during training, and (3) it enables better generalization by learning from diverse video content across multiple domains. The resulting framework demonstrates that effective dense video captioning can be achieved through careful utilization of existing data resources rather than requiring extensive new annotations.

\begin{sidewaysfigure}
    \centering
    \includegraphics[width=\textwidth]{figures/pws_network_architecture}
    \caption{Network architecture of the proposed model, PWS-DVC.
        PWS-DVC consists of two modules: caption localization and event captioning.
        The caption localization module takes a video and a caption as input and aims to localize which temporal region the caption corresponds to in the video.
        The event captioning module aims to generate a sentence describing the event given a video and its event.
        Both modules enable dense video captioning by alternatively training with video and ground-truth captions without supervision on events.}
    \label{fig:net_archi}
\end{sidewaysfigure}
% \begin{figure}[t]
%     \centering
%     \includegraphics[width=\textwidth]{figures/pws_network_architecture}
%     \caption{Network architecture of the proposed model, PWS-DVC.
%         PWS-DVC consists of two modules: caption localization and event captioning.
%         The caption localization module takes a video and a caption as input and aims to localize which temporal region the caption corresponds to in the video.
%         The event captioning module aims to generate a sentence describing the event given a video and its event.
%         Both modules enable dense video captioning by alternatively training with video and ground-truth captions without supervision on events.}
%     \label{fig:net_archi}
% \end{figure}

\section{PWS-DVC Framework}

\subsection{Overview}
Our proposed method, PWS-DVC, addresses the constraint of existing dense video captioning methods, which rely on specialized datasets for dense video captioning, such as ActivityNet Captions~\cite{Krishna2017-pw}.
Instead, we leverage weakly supervised dense video captioning to its maximum capacity by pretraining on readily accessible datasets for video clip-level captioning.
As depicted in Figure~\ref{fig:net_archi}, PWS-DVC comprises two distinct modules, namely the caption localization module and the event captioning module.
The module for captioning localization is responsible for determining the specific temporal interval in a video where a particular caption is positioned. On the other hand, the event captioning module is designed to construct a phrase that describes a video clip based on the provided video and event.
To address the difficulty of dense video captioning, we employ a two-step approach. Firstly, we pretrain the event captioning module using video clip-level datasets.
Subsequently, the two modules are trained in an alternating manner.

\begin{algorithm}
    \caption{Training procedure of PWS-DVC.}
    \begin{algorithmic}[1]
        \renewcommand{\algorithmicrequire}{\textbf{Input:}}
        \renewcommand{\algorithmicensure}{\textbf{Output:}}
        \REQUIRE $\mathcal{D}_\text{MSR-VTT}, \mathcal{D}_\text{ANetC}$ // datasets used for training
        \REQUIRE $\mathcal{C}, \mathcal{G}$ // event captioning and caption localization modules
        \REQUIRE $\left\{\boldsymbol{E}_j^{(a)}\right\}_{j=0}^{N_a}$ // anchors for training $\mathcal{G}$
        \REQUIRE $E^{(0.5,1)}$ // event that covers the entire video.
        \ENSURE  $\theta_\mathcal{C}$, $\theta_\mathcal{G}$ // trained parameters
        \\ Pretrain procedure
        \FOR {($\bm{V},\{C_i\}_{i=0}^{N_v}) \in \mathcal{D}_\text{MSR-VTT}$}
        \STATE $C = \textit{ChooseOne}(\{C_i\}_{i=0}^{N_v})$
        \STATE $C^\prime = \mathcal{C}(V, E^{(0.5,1)})$
        \STATE $\mathcal{L}_c = \textit{CrossEntropy}(C, C^\prime)$
        \STATE $\theta_\mathcal{C} = \theta_\mathcal{C} - \eta_\mathcal{C} \frac{\partial \mathcal{L}_c}{\partial \theta_\mathcal{C}}$
        \ENDFOR
        \\ Fine-tuning procedure
        \FOR {($\bm{V},\{C_i\}_{i=0}^{N_v}) \in \mathcal{D}_\text{ANetC}$}
        \STATE $C = \textit{ChooseOne}(\{C_i\}_{i=0}^{N_v})$
        \STATE $E^\prime, \text{confidence} = \mathcal{G}(V,C+\delta)$
        \STATE $C^\prime = \mathcal{C}(V,E^\prime)$
        \STATE $E^{\prime\prime} = \mathcal{G}(V,C^\prime)$
        \STATE $\{C_j\}_{j=0}^{N_a} = \{\mathcal{G}(V, E_j)\}_{j=0}^{N_a}$
        \STATE $\{\text{score}_j\}_{j=0}^{N_a}=\{\text{METEOR}(C,C_j)\}_{j=0}^{N_a}$
        \STATE $\text{best}=\text{argmax}_j\{\text{score}_j\}_{j=0}^{N_a}$
        \STATE $\mathcal{L}_c = \textit{CE}(C, C^\prime)$
        \STATE $\theta_\mathcal{C} = \theta_\mathcal{C} - \eta_\mathcal{C} \frac{\partial \mathcal{L}_c}{\partial \theta_\mathcal{C}}$
        \STATE $\mathcal{L}_l = \textit{MSE}(E^\prime, E^{\prime\prime}) + \textit{CE}(\text{confidence, \text{best}})$
        \STATE $\theta_\mathcal{G} = \theta_\mathcal{G} - \eta_\mathcal{G} \frac{\partial \mathcal{L}_l}{\partial \theta_\mathcal{G}}$
        \ENDFOR
        \RETURN $\theta_\mathcal{C}$, $\theta_\mathcal{G}$
    \end{algorithmic}
    \label{algo:training_procedure}
\end{algorithm}


\subsection{Caption Localization Module}
The goal of the sentence localization module is to accurately determine the temporal position of a sentence generated by the event captioning module, which is based on a specific segment of a video.
To achieve this objective, the event localization module analyzes the correlation between video frames and captions.
This task is accomplished by leveraging the temporal correlation of representations within the videos.

We adopt transformer encoder~\cite{Vaswani2017-sc} to encode both video and caption, enabling exploration of the temporal association between these two modalities through their respective representations, which is proven highly effective in capturing sequential data.
The process involves sequentially encoding video and caption using transformer encoders, resulting in the generation of video context memory and caption context memory, respectively.
The input is then subjected to an attention-pooling layer to rectify its temporal structure.
Finally, video and caption context memory integration is achieved by employing an attention network, allowing for the examination of video representations that align with the given fixed caption temporally.

We formulate this process as follows.
In the context of the transformer model, the initial step involves the application of positional encoding to both a sequence of video frames and a sequence of words inside a sentence, as illustrated in Figure~\ref{fig:net_archi}.
\begin{equation}
    \bm{H}_v^0 = \text{PE} (\text{Linear} \left( \bm{F} \right) )
    \label{eq:linear_transformation}
\end{equation}
The positional encoding function, denoted as PE($\cdot$), is applied to the frame-level video feature $\bm{F}$, which represents the input video $v$ or the embedding vector of the input caption.
The frame-level video feature is based on the C3D model proposed by~\cite{Tran2015-uq}.

The video and caption encoding components are based on the transformer encoder, which comprises a multiple of $N$ layers.
The composition of each layer consists of self-attention and feed-forward modules, with a residual connection and layer normalization appended after each module.
The output $\bm{H}^{l+1}$ of the transformer encoder is obtained in a single cycle, using the input $\bm{H}^l$.
The final output of the last cycle, which corresponds to the last layer, is denoted as $\bm{H}$.
The subsequent pair of equations can be employed to delineate encoding based on transformers.
\begin{equation}
    \bm{H}^{l}_{self} = \Psi \left(H^l + \text{MA}\left( \bm{H}^l, \bm{H}^l, \bm{H}^l \right) \right)
    \label{eq:video_encoder_self_attention}
\end{equation}
\begin{equation}
    \bm{H}^{l+1} = \Psi \left( \text{FFN}\left(\bm{H}^{l}_{self}\right) + \bm{H}^{l}_{self} \right)
    \label{eq:video_encoder_ffn}
\end{equation}
\begin{equation}
    \text{FFN}(x) = \text{max} \left(0, x\bm{W}_1 + b_1 \right)\bm{W}_2 + b_2
    \label{eq:video_encoder_feed_forward}
\end{equation}
\begin{equation}
    \text{MA}(\bm{Q}, \bm{K}, \bm{V}) = \text{Cat}(\{\text{ATT}_i(\bm{Q}, \bm{K}, \bm{V})\}_{i=1}^h)W_O
\end{equation}
\begin{equation}
    \text{ATT}(\bm{Q}, \bm{K}, \bm{V}) = \text{softmax}( \frac{(\bm{QW}_Q)(\bm{K}\bm{W}_K)^T}{\sqrt{d_m}} ) (\bm{VW}_V)
    \label{eq:video_encoder_att}
\end{equation}
where, $\Psi\left(\cdot\right)$ represents the layer normalization function.
$\text{Cat}(\cdot)$ denotes vector concatenation and $\bm{W}_*$ are trainable parameters.
We repeat the above process as many times as the number of encoder layers.
The output of the video encoder is the output of the last layer $\bm{H}_v$ and $\bm{H}_c$, for the video and the caption, respectively.

The attention-based feature fusing process is a mechanism used to align and fuse video context memory and caption context memory in order to determine the temporal region in a video that corresponds to the caption.
This process involves several steps.
To start, we have two sets of context memories: the video context memory represented as $\bm{H}_v$ and the caption context memory represented as $\bm{H}_c$.
These context memories contain information about the video and caption, respectively.

The first step in attention-based feature fusing is to achieve temporal dimension matching between these two sets of memories.
This means that we want to make sure that both memories have the same dimensions and can be effectively compared.
To achieve this matching, an attention pooling layer is employed.
This layer computes attention weights for each element in the memories, effectively giving importance scores to each element.

The formula used to calculate the attention weights is given as follows:
\begin{equation}
    \bm{a}_{i} = \frac{e^{w \cdot \bm{h}_i}}{\sum_{j=1}^{K}e^{w \cdot \bm{h}_j}}
\end{equation}
where, $\bm{a}_{i}$ represents the attention weight assigned to each element of memory $H$, denoted as $\bm{h}_i$.
$w$ represents the attention weights, which are learnable parameters in the model.
$K$ represents the number of dimensions in the output vector.
This value is determined by the hidden dimension of the transformer hyperparameter ($K$ corresponds to the hidden dimension).
Once we have computed these attention weights, we proceed to fuse the memories.
The pooled memory, denoted as $\bm{H}_{pooled}$, is calculated by taking a weighted sum of the elements in the memory, where the weights are determined by the attention scores:
\begin{equation}
\bm{H}_{pooled} = \sum_{i=1}^{N}\bm{a}_i \cdot \bm{h}_i
\end{equation}
where, $\bm{H}_{pooled}$ represents the fused memory that combines information from both the video and caption context memories.
Each element of this pooled memory is a weighted combination of the corresponding elements in the original memories, with the weights determined by the attention scores.

Now, with the pooled video context memory $\bm{H}_{v,\text{pooled}}$ and caption context memory $\bm{H}_{c,\text{pooled}}$, we have effectively aligned and fused the information from both modalities.
The final step in attention-based feature fusing is to compute the importance of the video context memory to the caption context memory.
This importance score helps identify the temporal region in the video that corresponds to the caption, allowing the model to focus on relevant video features for localization.

\iffalse
In order to achieve temporal dimension matching between the video context memory $\bm{H}_v$ and caption context memory $\bm{H}_c$, we employ an attention pooling layer.
This allows us to fuse the two memories effectively.
\begin{equation}
    \hleq
    {\bm{a}_{i} = \frac{e^{w \cdot \bm{h}_i}}{\sum_{j=1}^{K}e^{w \cdot \bm{h}_j}}}
\end{equation}
In the given equation, $\bm{a}_{i}$ represents the weight assigned to each element of memory $H$, denoted as $\bm{h}_i$.
$w$ represents the attention weights, and $K$ represents the number of dimensions in the output vector.
\hl
{$K$ corresponds to the hidden dimension of the transformer hyperparameter (refer to section~\ref{subsec:experiment_setup} for more information).}
The pooled memory is calculated as follows:
\begin{equation}
    \bm{H}_{pooled} = \sum_{i=1}^{N}\bm{a}_i \cdot \bm{h}_i
\end{equation}

Given a pooled video context memory $\bm{H}_{v,\text{pooled}}$ and a caption context memory $\bm{H}_{c,\text{pooled}}$, we compute the importance of the video context memory to the caption context memory to find the temporal region where the caption corresponds to the video.
We call this process attention-based feature fusing.
This allows us to learn which video features the caption should be interested in for localization.
\fi

The last step in caption localization involves localizing the temporal region in the video that corresponds to the caption.
This is done using anchor classification and delta regression techniques.
The use of simple regression, as employed in the existing methods, may aid in achieving this task.
However, it should be noted that this approach is prone to encountering local minima.
To address this issue, the localization process is divided into two distinct steps: classification and refinement.
The classification step aims to select the most appropriate one among several predefined anchors, and the refinement step aims at regression to compensate for the selected anchor.

To do this, we adopt two fully connected layers, $\text{FC}_{\text{cls}}$ and $\text{FC}_{\text{reg}}$.
We first identify the best anchor from a set of predefined anchors $S=\{e_1, e_2, \cdots, e_{N_a-1}\}$.
\begin{equation}
    \{o_i\}_{i=0}^{N_a} = \{ \text{FC}_{\text{cls}} \left( e_i \right) \}_{i=0}^{N_r}
\end{equation}
where $o_i$ means the classification logit for the $i$-th anchor.
Then we formulate the index of best anchor $i^{(\text{best})}$ as follows.
\begin{equation}
    i^{(\text{best})} = \arg \max_{i} \{\text{MeteorScore}(o_i)\}_{i=0}^{N_a}
\end{equation}
where, the $\text{MeteorScore}(\cdot)$ function calculates the METEOR~\cite{Banerjee2005-zo} score of the caption generated for the anchor.
We regress $\Delta w$ and $\Delta m$ to refine the final best anchor.
\begin{equation}
    \Delta w, \Delta m = \text{FC}_{\text{reg}} (e_{i^{(\text{best})}})
\end{equation}
We formulate the anchor that is finally selected, the anchor with the best score is $e^{(\text{best})}=(w_{i^{(\text{best})}}+\Delta w,m_{i^{(\text{best})}}+\Delta m)$.
$w_{i^{(\text{best})}}$ and $m_{i^{(\text{best})}}$ are the width and center for the best anchor, respectively.

\subsection{Event Captioning Module}
The objective of the event captioning module is to provide a concise and informative sentence that describes an event, based on a video and its corresponding temporal region.
To achieve this, we follow the established framework of the vanilla transformer~\cite{Vaswani2017-sc}.
Specifically, we use a two-step process: first, we take a whole video and an associated event as inputs; second, we apply a mask to the temporal region of the video feature that corresponds to the event, and feed it into the transformer encoder. The last step involves allowing the visual context memory output, derived from the transformer encoder, to undergo cross-attention with the transformer's decoder.
This process facilitates the generation of the relevant caption.

We apply an event mask to the video and let $\bm{H}_v$ be the feature of the video after applying positional encoding, and the above process can be expressed as the following equations.
\begin{equation}
    \bm{H}^{l}_{self} = \Psi \left(H_v^l + \text{MA}\left( \bm{H}_v^l, \bm{H}_v^l, \bm{H}_v^l \right) \right)
\end{equation}
We denote the output of the last encoder layer, which is the video context memory, as $\bm{H}_v$, the decoding process of the transformer be represented as follows.
\begin{equation}
    \bm{H}^l_{cross} = \Psi \left( H_v + \text{MA}\left( \bm{H}^{l}_{self}, \bm{H}_v, \bm{H}_v \right) \right)
\end{equation}
\begin{equation}
    \bm{H}_{cross}^{l+1} = \Psi \left( \bm{H}^{l}_{cross} + \text{FFN} \left( \bm{H}^{l}_{cross} \right) \right)
\end{equation}
We repeat the aforementioned process, where $\bm{H}_{cross}^{l+1}$ is set as $\bm{H}_c$ for the final layer.
$\bm{H}_c$ is the output of the transformer decoder, which classifies a word from one of the vocabulary as the last to pass through the linear transformation as shown in Figure~\ref{fig:net_archi}.

\subsection{Training Process}
As described in Algorithm~\ref{algo:training_procedure}, the PWS-DVC training process involves two steps: pretraining and fine-tuning.
During pretraining, the focus is on the event captioning module.
Developing a solid language model can be a challenging task, as it requires a substantial text corpus.
Therefore, we use a pretraining approach to generate descriptive sentences that describe the corresponding videos.
This approach helps to achieve more accurate event localization.
For pretraining, we use widely available video clip-level datasets such as MSR-VTT~\cite{Xu2016-ti} and MSVD~\cite{Chen2011-ai}.
These datasets are ideal for pretraining the event captioning module because they are already trimmed to a short event level.
To obtain video features, we pass a video through a C3D~\cite{Tran2015-uq} backbone, resulting in C3D features in 16 non-overlapped frames of the video.
We then pass these video features through the event captioning module to train with the following loss function.
\begin{equation}
    \mathcal{L}_c=-\sum_{t=1}^{T_c} \bm{c}_t \cdot \log \left(\hat{\bm{c}}_t \mid \bm{c}_0: \bm{c}_{t-1}\right)
    \label{eq:loss_ec}
\end{equation}
where, $\hat{\bm{C}}=\left\{\hat{\bm{c}}_i\right\}_{i=0}^{T_c}$ means the ground-truth caption, and $\bm{C}=\left\{\bm{c}_i\right\}_{i=0}^{T_c}$ means the prediction output by the event captioning module.

Subsequently, we proceed with the process of fine-tuning using datasets such as the ActivityNet Captions dataset, which is specifically designed for dense video captioning tasks.
The event captioning and caption localization modules are trained in an alternating manner.
Although our method for weakly supervised dense video captioning has the potential to improve over multiple rounds of training, we focus on a single-round iteration, as depicted in Figure~\ref{fig:net_archi}.
The equation denoted as Equation~(\ref{eq:loss_ec}) is employed for the event captioning module, and the following loss function is taken into account for caption localization.
\begin{equation}
    \mathcal{L}_l=\left[ \left( m^{\prime} - m^{\prime \prime}\right)^2 + \left( w^{\prime} - w^{\prime \prime} \right)^2 \right] + \left[ -\sum_{i=0}^{N_a} y_i \log p_i \right]
    \label{eq:loss_cl}
\end{equation}
The first term is the L2 norm for regression on the center and width of the event, and the second term is the cross-entropy loss for anchor classification.
Specifically, $m^{\prime}$ and $w^{\prime}$ refer to the center and length of $E^{\prime}$, and $m^{\prime \prime}$ and $w^{\prime \prime}$ refer to the centre and length of $E^{\prime \prime}$ in Figure~\ref{fig:net_archi}.
For classification, we define a one-hot label $y = \{y_1, \cdots, y_{N_a} \}$ for the predefined anchors.
If $y_i$ is 1, it indicates that the $i$-th anchor is the best, otherwise it is 0.
$p = \{p_1, \cdots, p_{N_a} \}$ is the prediction of the caption localization module.


\subsection{Inference}
Since the PWS-DVC model lacks a direct training mechanism for dense video captioning, we explain the inference procedure.
To conduct inference on a video $V$, we begin by selecting a set of random event anchors denoted as $\{E_i^{(a)}\}_{i=0}^{N_r}$, where $N_r$ is set to 15.
Specifically, $N_r$ random numbers are selected from the interval $[0, 1]$ to represent the width and center of the anchors.
Next, we combine them to obtain anchors consisting of pairs of $N_r$ widths and centers.
After performing a single iteration for each event anchor, we obtain a set of refined events denoted as $\{S_i\}_{i=0}^{N_r}$.
We filter the events based on their values with high temporal Intersection over Union (tIoU), retaining only those that meet the validity criteria.
Finally, we input the filtered events into the event captioning module to provide captions for each event.

\section{Experimental Validation}

\subsection{Experimental Setup}
\label{subsec:experiment_setup}

\textbf{Datasets.}
The MSR-VTT dataset~\cite{Xu2016-ti} is used to pretrain the event captioning module.
This comprehensive video benchmark dataset contains 10,000 web video clips, each accompanied by 20 annotations written by humans.
A total of 200,000 clip-sentence pairings were used as samples for pretraining.
Additionally, we employ temporal augmentation techniques to increase the number of samples from 200K to 1M.

For dense video captioning, we use the ActivityNet Captions~\cite{Krishna2017-pw} dataset, which comprises 20K untrimmed videos capturing various human activities.
On average, videos have a duration of 120 seconds and contain an average of 3.7 events.
While other datasets exist for dense video captioning, such as YouCook2~\cite{Zhou2018-eq} and ViTT~\cite{Huang2020-as}, we focus exclusively on the ActivityNet Captions dataset for the purpose of comparing with current weakly dense video captioning algorithms.

\textbf{Evaluation metrics.}
The publicly accessible evaluation code supplied by the ActivityNet Captions Challenge is utilized in our study\footnote{\url{https://github.com/ranjaykrishna/densevid_eval}}.
The evaluation metrics employed in this paper include recall and precision for event localization, as well as METEOR~\cite{Banerjee2005-zo}, CIDEr~\cite{Vedantam2015-ma}, and BLEU~\cite{Papineni2002-sn} for sentence evaluation.
The computation of the captioning score for a generated event and sentence pair is contingent upon the presence of an overlap greater than the specified threshold with any ground-truth events.
This score is determined by comparing the related sentence from the ground-truth data.
Alternatively, if the conditions are not met, the score is assigned a value of zero.

We adopt SODA~\cite{Fujita2020-ob} as an additional evaluation metric.
Recently, SODA has been introduced for evaluating multi-sentence captioning tasks, particularly in the context of dense video captioning.
The SODA metric considers the coherence and narrative flow between sentences.
It is derived from the METEOR metric, but penalizes the use of redundant words, phrases, and sentences.
This feature enhances its ability to resist over-evaluation, as it discourages inflated scores through excessive repetition.

\textbf{Implementation details.}
The event captioning module and caption localization module of PWS-DVC are based on the transformer architecture.
We follow the hyper-parameters of the original transformer~\cite{Vaswani2017-sc}, including a model dimension of 512, 8 encoders and decoders, and a feed-forward dimension of 2048.
We set the dropout rate to 0.1.

To train PWS-DVC, we use the AdamW~\cite{Loshchilov2017-sm} optimizer in both the pretraining and fine-tuning stages.
In the pretraining stage, we gradually decrease the learning rate according to a cosine function with a warm-up period that increases up to 5e-4.
In the fine-tuning stage, we train for 30 epochs using a batch size of 8 with a learning rate of 1e-4.

The examination of various hyperparameters' influence on the model is crucial in deep learning frameworks.
Our extensive exploration indicates that PWS-DVC is robust to different hyperparameters, meaning that the improvement in performance with better hyperparameter settings is minimal.
Instead of aiming for improved outcomes, we adhere to widely adopted settings to ensure fairness when comparing with baseline methods.


\subsection{Performance Comparison}
We conduct extensive comparisons with existing state-of-the-art methods to demonstrate the effectiveness of PWS-DVC. Besides the dense video captioning benchmark dataset, ActivityNet Captions, we also compare the performance with video-clip captioning methods and compare the performance of caption localization with temporal action localization models.

\textbf{Dense video captioning.}
Table~\ref{tab:performance} compares the dense video captioning performance on the ActivityNet Captions validation set.
The results show that PWS-DVC achieves the best performance on weakly supervised dense video captioning methods for METEOR, CIDEr, ROUGE, and BLEU, and comparable performance on fully supervised methods.
Specifically, PWS-DVC achieves state-of-the-art performance on almost all metrics compared to the weakly supervised approaches MUTAN, WS-DEC, and EC-SL.
These results suggest that PWS-DVC's pretraining approach contributes to both localization and captioning by improving the performance of the event captioning module.
Table~\ref{tab:performance} also compares PWS-DVC with several methods trained under weakly supervised as well as fully supervised conditions.
PWS-DVC performs comparably to them despite being trained without information about ground-truth events.

\newcommand{\mc}[1]{\multicolumn{#1}}
\begin{sidewaystable}
% \begin{table*}[tp]
    \centering
    \caption{Dense event captioning performances (\%) of the proposed method and state-of-the-art methods on the ActivityNet Captions validation set. ``WS'' denotes ``weakly supervised''.}
    \begin{tabular}{l|c|c|cccccccc}
        \hline
        \mc{1}{c|}{Method}         & WS     & Extra dataset & \mc{1}{c}{METEOR} & \mc{1}{c}{CIDEr} & \mc{1}{c}{ROUGE} & \mc{1}{c}{BLEU@1} & \mc{1}{c}{BLEU@2} & \mc{1}{c}{BLEU@3} & \mc{1}{c}{BLEU@4} & \mc{1}{c}{SODA\_c} \\ \hline
        DCE~\cite{Krishna2017-pw}  & \xmark & \xmark        & 4.82              & 17.29            & -                & 17.95             & 7.69              & 3.86              & 2.20              \\
        DVC~\cite{Li2018-ll}       & \xmark & \xmark        & 6.93              & 12.61            & -                & 12.22             & 5.72              & 2.27              & 0.73              \\
        Bi-SST~\cite{Wang2018-ap}  & \xmark & \xmark        & 9.60              & 12.68            & 19.10            & 18.99             & 8.84              & 4.41              & 2.30              \\
        MT~\cite{Zhou2018-zu}      & \xmark & \xmark        & 9.56              & -                & -                & -                 & -                 & 4.76              & 2.23              \\
        SDVC~\cite{Mun2019-ap}     & \xmark & \xmark        & 8.82              & 30.68            & -                & 17.93             & 7.99              & 2.94              & 0.93              \\
        PDVC~\cite{Wang2021-xe} & \xmark & \xmark &       7.50 & 47.26 & 2.64 & - & - & - & 1.65 & 5.26 \\
        E2ESG~\cite{Zhang2022-ni} & \xmark & \xmark &       7.33 & 26.92 & - & - & - & - & 1.45 & 5.29 \\
        Vid2Seq~\cite{Yang2023-fm} & \xmark & \xmark &       8.50 & 30.10 & - & - & - & - & - & 5.80 \\
        \hline
        MUTAN~\cite{Rahman2019-rp} & \cmark & \xmark        & 4.93              & 13.79            & 10.39            & 10.00             & 4.20              & 1.85              & 0.90              \\
        WS-DEC~\cite{Duan2018-qf}  & \cmark & \xmark        & 6.30              & 18.77            & 12.55            & 12.41             & 5.50              & 2.62              & 1.27              \\
        EC-SL~\cite{Chen2021-sv}   & \cmark & \xmark        & 7.49              & 21.21            & 13.02            & \textbf{13.36}    & 5.96              & 2.78              & 1.33              \\ \hline
        PWS-DVC                    & \cmark & \cmark        & \textbf{8.22}     & \textbf{28.91}   & \textbf{13.80}   & 12.92             & \textbf{6.38}     & \textbf{3.11}     & \textbf{1.87} & \textbf{5.45}     \\ \hline
    \end{tabular}
    \label{tab:performance}
% \end{table*}
\end{sidewaystable}

\begin{table}[tp]
    \centering
    \caption{Temporal event localization performances of PWS-DVC and state-of-the-art methods on the ActivityNet Captions dataset.
        The top-1 recall is quantified by evaluating the overlap between the predicted temporal event and the ground truth across various temporal IoU thresholds.
        The abbreviation ``WS'' is used to represent the term ``weakly supervised''.
    }
    \begin{tabular}{l|c|c|ccc}
        \hline
        \multicolumn{1}{c|}{Method} & WS     & Pretrain & tIoU=0.1 & tIoU=0.3 & tIoU=0.5 \\ \hline
        CTRL~\cite{Gao2017-ij}      & \xmark & \xmark   & -        & 47.43    & 29.01    \\
        ABLR~\cite{Yuan2019-zc}     & \xmark & \xmark   & 73.30    & 55.67    & 36.79    \\
        2D-TAN~\cite{Zhang2020-rm}  & \xmark & \xmark   & -        & 58.75    & 44.05    \\
        FIAN~\cite{Qu2020-dg}       & \cmark & \xmark   & -        & 64.10    & 47.90    \\ \hline
        WSLLN~\cite{Gao2019-kd}     & \cmark & \xmark   & 75.40    & 42.80    & 22.70    \\
        LCGB~\cite{Chen2020-fu}     & \cmark & \xmark   & 74.20    & 44.30    & 23.60    \\
        SCN~\cite{Duan2018-qf}      & \cmark & \xmark   & 71.48    & 47.23    & 29.22    \\ \hline
        WS-DEC~\cite{Duan2018-qf}   & \cmark & \xmark   & 62.71    & 41.98    & 23.34    \\
        EC-SL~\cite{Chen2021-sv}    & \cmark & \xmark   & 68.48    & 44.29    & 24.16    \\ \hline
        PWS-DVC                     & \cmark & \cmark   & 70.01    & 51.94    & 27.70    \\ \hline
    \end{tabular}
    \label{tab:localization}
\end{table}


\begin{table}[tp]
    \centering
    \begin{tabular}{l|ccc}
        \hline
        \multicolumn{1}{c|}{Method}                   & \multicolumn{1}{c}{BLEU@4} & \multicolumn{1}{c}{CIDEr} & \multicolumn{1}{c}{METEOR} \\ \hline
        FSTA~\cite{Liu2018-pg}                        & 39.80                      & 41.1                      & 26.50                      \\
        OA-BTG~\cite{Zhang2019-kb}                    & 41.40                      & 46.9                      & 28.20                      \\
        MGSA~\cite{Chen2019-po}                       & 42.40                      & 47.5                      & 27.60                      \\
        POS+CG~\cite{Wang2019-dh}                     & 42.00                      & 48.7                      & 28.20                      \\
        POS+VCT~\cite{Hou2019-ox}                     & 42.30                      & 49.1                      & 29.70                      \\
        SAM-SS~\cite{Chen2020-di}                     & 43.80                      & 53.2                      & 28.90                      \\
        ORG-TRL~\cite{Zhang2020-rl}                   & 43.60                      & 50.9                      & 28.80                      \\
        STraNet~\cite{Zhang2020-rl}                   & 43.60                      & 50.9                      & 28.80                      \\
        VNS-GRU~\cite{Chen2020-iy}                    & 45.30                      & 53.0                      & 29.90                      \\
        ARB~\cite{Yang2021-xf} + ACL~\cite{Li2022-uh} & 42.60                      & 51.3                      & 28.90                      \\
        MFVCG~\cite{Bhooshan2022-ex}                  & 44.90                      & 52.2                      & 29.80                      \\ \hline
        PWS-DVC                                       & 41.20                      & 49.3                      & 26.99                      \\ \hline
    \end{tabular}
    \caption{Comparison to state-of-the-art methods on MSR-VTT for video clip captioning.}
    \label{tab:clip-captioning}
\end{table}

% \begin{table*}[tp]
\begin{sidewaystable}
    \centering
    \begin{tabular}{c|l|cccc}
        \hline
        \# & \multicolumn{1}{c|}{Method}                              & METEOR & CIDEr & ROUGE & BLEU@4 \\ \hline
        1  & PWS-DVC without pretraining                              & 5.49   & 13.77 & 10.02 & 0.72   \\
        2  & PWS-DVC pretrained on ActivityNet Captions               & 7.28   & 20.59 & 12.71 & 1.35   \\
        3  & PWS-DVC pretrained on MSR-VTT                            & 8.05   & 27.11 & 13.17 & 1.80   \\ \hline
        4  & PWS-DVC pretrained on MSR-VTT with temporal augmentation & 8.22   & 28.91 & 13.80 & 1.87   \\ \hline
    \end{tabular}
    \caption{The results of ablation study.}
    \label{tab:ablation}
% \end{table*}
\end{sidewaystable}

\textbf{Caption localization.}
We then evaluate the performance of caption localization on the validation set of ActivityNet Captions.
We compare the weakly supervised dense video captioning methods WS-DEC and EC-SL with state-of-the-art models for temporal action localization tasks, such as CTRL and ABLR.
Table~\ref{tab:localization} shows that the PWS-DVC approach achieves superior accuracy in caption localization compared to existing weakly supervised dense video captioning methods WS-DEC and EL-SL.
Despite the fact that existing temporal action localization methods are trained on ground-truth events, the caption localization performance of PWS-DVC is comparable.
These results demonstrate that PWS-DVC's approach is not only effective in improving the event captioning module, but also in improving caption localization.

\textbf{Video clip captioning.}
The event captioning module of PWS-DVC generates a single sentence from a video clip, which is itself the purpose of video-clip captioning.
We demonstrate the effectiveness of our event captioning module not only for dense video captioning, but also by comparing it with existing video clip captioning methods on the MSR-VTT video clip dataset.
Table~\ref{tab:clip-captioning} shows the performance comparison of PWS-DVC's event captioning module with state-of-the-art video clip captioning methods.
While video clip captioning methods have various advantages such as multimodal, pretraining, and object detection, PWS-DVC follows the architecture of a vanilla transformer in terms of model architecture and modality.
Nevertheless, the event captioning module of PWS-DVC shows comparable performance to existing state-of-the-art video clip captioning methods.

\begin{sidewaysfigure}
    \centering
    \includegraphics[width=\textwidth]{figures/pws_exp_qualitative}
    \caption{Examples of the output of PWS-DVC for two videos from the ActivityNet Captions validation set.}
    \label{fig:qualitative}
\end{sidewaysfigure}

% \begin{figure*}[t]
%     \centering
%     \includegraphics[width=\textwidth]{figures/pws_exp_qualitative}
%     \caption{Examples of the output of PWS-DVC for two videos from the ActivityNet Captions validation set.}
%     \label{fig:qualitative}
% \end{figure*}

\subsection{Qualitative Results}
Figure~\ref{fig:qualitative} presents the qualitative outcomes of PWS-DVC.
We select two videos from the validation set of ActivityNet Captions, and show their corresponding events and captions together.
Through the examination of these two instances, it is demonstrated that PWS-DVC has the capability to effectively generate events and captions even in the absence of ground-truth events.
Furthermore, PWS-DVC produces descriptive sentences by utilizing the event captioning module that has been pretrained on a video clip dataset.
Additionally, it accurately identifies the location of each sentence inside the video scene.
In the first video, PWS-DVC produces sentences that incorporate certain keywords, such as ``green field,'' ``running,'' and ``lacrosse,'' while also accurately identifying the precise location within the video scene for each sentence.
This holds for the second video as well.

Note that the outcomes of PWS-DVC do not align with the ground-truth events and captions.
However, this is a reasonable observation considering the inherent non-deterministic character of the dense video captioning task, as depicted in Figure~\ref{fig:nondeterminism}.

\subsection{Ablation Study}
To demonstrate the efficacy of pretraining the event captioning module, which serves as the core approach of PWS-DVC, we conduct an intensive ablation experiment involving different pretraining.
We train four different variants of PWS-DVC, each with varying levels of pretraining: (i) PWS-DVC without pretraining, (ii) PWS-DVC pretrained using only the ActivityNet Captions dataset without any extra dataset, (iii) PWS-DVC pretrained on MSR-VTT but without temporal augmentation, and (iv) PWS-DVC pretrained on MSR-VTT with temporal augmentation.
We present the performance of the four distinct PWS-DVC models on the ActivityNet Captions validation set in Table~\ref{tab:ablation}.
It is evident that models that have undergone pretraining on ActivityNet Captions or MSR-VTT exhibit superior performance compared to model \#1.
This observation indicates that pretraining is highly effective in the context of weakly supervised dense video captioning approaches.
Furthermore, it is evident that the incorporation of extra datasets, such as MSR-VTT, with the ActivityNet Captions dataset, yields enhanced performance outcomes.
This is attributed to the augmented diversity of the training sample.
By the same principle, the temporal augmentation serves to enhance the diversity of the training dataset, hence resulting in model \#4 attaining the highest level of performance.



\section{Discussion}
\label{sec:discussion}

\subsection{Strengths and Contributions of PWS-DVC}

The PWS-DVC framework demonstrates several key advantages that address fundamental challenges in dense video captioning while reducing the dependency on expensive temporal annotations.

\textbf{Effective Leveraging of Large-Scale Data.}
Our approach successfully demonstrates that dense video captioning models can be effectively initialized using readily available video-caption datasets such as MSR-VTT~\cite{Xu2016-ti} and HowTo100M~\cite{Miech2019-hk}, without requiring precise event-level annotations.
This represents a significant departure from traditional approaches that rely exclusively on manually annotated datasets like ActivityNet Captions~\cite{Krishna2017-pw}. The pretraining strategy enables the model to learn robust visual-linguistic representations that transfer effectively to the dense captioning task, resulting in improved performance compared to training from scratch.

\textbf{Robust Language Model Development.}
The pretraining of the event captioning module on large-scale clip-level data addresses a critical limitation in weakly supervised dense video captioning: the chicken-and-egg problem where insufficient sentences generated by the captioning module hinder accurate localization~\cite{Duan2018-qf,Shen2017-gx}. By ensuring that the captioning module produces coherent and descriptive sentences about video content, PWS-DVC facilitates more accurate localization in the subsequent fine-tuning stage.

\textbf{Training Stability and Generalization.}
The experimental results demonstrate that pretraining significantly improves training stability in weakly supervised scenarios. Unlike fully supervised methods that can overfit to specific annotation patterns, PWS-DVC exhibits better generalization capabilities across different video domains, as evidenced by consistent performance improvements on both ActivityNet Captions and YouCook2 datasets.

\subsection{Limitations and Challenges}

Despite its effectiveness, PWS-DVC faces several limitations that warrant careful consideration.

\textbf{Domain Gap Between Pretraining and Target Data.}
The domain shift between clip-level captioning datasets and dense video captioning scenarios presents ongoing challenges. While MSR-VTT contains diverse video content, the nature of single-event clips differs fundamentally from the multi-event untrimmed videos in ActivityNet Captions~\cite{Krishna2017-pw}. This domain gap can limit the effectiveness of knowledge transfer, particularly for videos containing complex temporal structures or domain-specific activities.

\textbf{Limited Event Temporal Modeling.}
The current framework focuses primarily on improving caption quality through pretraining but provides limited enhancements to temporal event localization capabilities. The alternating training strategy between event captioning and caption localization modules~\cite{Chen2021-sv} remains susceptible to convergence issues when dealing with complex temporal patterns or highly overlapping events.

\textbf{Dependency on Data Quality.}
The effectiveness of PWS-DVC is inherently dependent on the quality and diversity of pretraining data. Noisy captions, domain-specific vocabulary, or biased content in large-scale datasets can negatively impact the learned representations and subsequently affect dense captioning performance~\cite{Miech2019-hk}.

\subsection{Comparison with Alternative Approaches}

\textbf{Versus Fully Supervised Methods.}
While PWS-DVC achieves competitive performance compared to fully supervised approaches~\cite{Wang2021-zi,Zhou2018-zu}, it demonstrates particular advantages in scenarios with limited annotated data. The pretraining strategy provides a strong initialization that enables effective learning even with reduced supervision, making it particularly valuable for domains where temporal annotation is expensive or impractical.

\textbf{Versus Other Weakly Supervised Methods.}
Compared to existing weakly supervised approaches~\cite{Duan2018-qf}, PWS-DVC addresses the language model quality issue more directly through systematic pretraining. This targeted approach to improving caption generation capabilities represents a more principled solution than relying solely on architectural modifications or training procedure adjustments.

\textbf{Versus Vision-Language Pretraining Methods.}
Recent vision-language pretraining approaches such as CLIP~\cite{Radford2021-kx} and VideoCLIP have shown promise for video understanding tasks. However, these methods typically focus on global video-text alignment rather than the fine-grained temporal localization required for dense captioning. PWS-DVC's task-specific pretraining strategy proves more effective for the particular challenges of event-level video description.

\subsection{Practical Implications and Applications}

\textbf{Scalability to New Domains.}
The PWS-DVC framework offers significant practical advantages for deploying dense video captioning systems in domains where temporal annotations are scarce or unavailable. The ability to leverage existing video-caption datasets for initialization enables rapid adaptation to new video domains with minimal annotation requirements.

\textbf{Cost-Effective Annotation Strategies.}
Our approach suggests new directions for cost-effective data collection strategies in video understanding. Rather than investing heavily in precise temporal annotations, practitioners can focus on collecting high-quality video-caption pairs at the clip level, which are significantly cheaper and faster to annotate.

\textbf{Integration with Modern Architectures.}
The pretraining principles demonstrated in PWS-DVC are compatible with modern transformer-based architectures~\cite{Vaswani2017-sc} and can be extended to leverage recent advances in vision-language modeling. The framework provides a foundation for incorporating more sophisticated pretraining objectives and multi-modal learning strategies.

\subsection{Future Research Directions}

\textbf{Multi-Scale Pretraining.}
Future work could explore multi-scale pretraining strategies that progressively increase temporal complexity, starting from clip-level captions and gradually incorporating longer temporal contexts. This hierarchical approach could better bridge the gap between single-event and multi-event video understanding.

\textbf{Cross-Modal Enhancement.}
Incorporating audio and text modalities during pretraining could further improve the robustness of learned representations. Recent work on audio-visual learning~\cite{Akbari2021-ih} suggests promising directions for enhancing dense video captioning through multi-modal pretraining.

\textbf{Self-Supervised Learning Integration.}
Combining PWS-DVC with self-supervised learning objectives could reduce the dependency on human-annotated captions while maintaining effective knowledge transfer. Techniques such as masked language modeling applied to video-text pairs could provide additional supervision signals during pretraining.

\textbf{Adaptive Fine-Tuning Strategies.}
Developing adaptive fine-tuning strategies that account for domain-specific characteristics could further improve the effectiveness of knowledge transfer from pretraining to target tasks. This includes developing methods to identify and preserve the most relevant learned representations while adapting to new video domains.

The PWS-DVC framework represents a significant step toward practical and scalable dense video captioning systems. By demonstrating the effectiveness of task-specific pretraining in weakly supervised scenarios, this work opens new avenues for reducing annotation requirements while maintaining competitive performance in complex video understanding tasks.